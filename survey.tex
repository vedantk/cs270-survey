\documentclass{article}

\usepackage{amsmath}
\usepackage{amssymb}
\usepackage{enumerate}

\title{Spectral Graph Sparsification}
\author{Vedant Kumar}

\begin{document}
\maketitle

\section*{Abstract}

Graph sparsification produces sparse approximations of input graphs.  In
this paper we survey major results in spectral sparsification, a special
case in which important algebraic properties of graphs are preserved.  We
provide intuition and rigour by explaining a key result: that every graph
has a spectral sparsifier with $O(|V|\log|V|)$ edges.  We finish by
discussing applications of spectral sparsification, such as computing
electrical flows and finding min $s$-$t$ cuts.

\section*{Introduction}

Research in graph sparsification has been motivated by the fact that sparse
graphs are easier to compute with than dense graphs. 

\section*{References}

\begin{enumerate}[1.]
    \item Achlioptas, D., Mcsherry, F. Fast computation of low-rank matrix
approximations. 2 (2007), 9.

    \item Batson, J.D., Spielman, D.A., Srivastava, N.  Twice-Ramanujan
sparsifiers. 6 (2012), 1704 –1721.

    \item Benczúr, A.A., Karger, D.R.  Approximating s-t minimum cuts in O(n 2
) time. In (1996), 47–55.

    \item Chandra, A.K., Raghavan, P., Ruzzo, W.L., Smolensky, R., Tiwari, P.
The electrical resistance of a graph captures its commute and cover times. In
(1989), ACM, New York, NY, USA, 574–586.

    \item Cheeger, J. A lower bound for smallest eigenvalue of Laplacian. In
(1970), Princeton University Press, 195–199.
\end{enumerate}

\end{document}
