\documentclass{article}

\usepackage{amsmath}
\usepackage{amssymb}
\usepackage{enumerate}
\usepackage[a4paper]{geometry}

\title{Spectral Graph Sparsification}
\author{Vedant Kumar}

\begin{document}
\maketitle

\newcommand \textlcsc[1]{\textsc{\MakeLowercase{#1}}}

\section*{Abstract}

Graph sparsification algorithms find sparse approximations of graphs. In
this paper we survey major results in spectral sparsification, a special
case in which important structural and algebraic properties of graphs are
preserved. We explain and provide intuition for a key result: that w.h.p
every graph has a spectral sparsifier with $O(|V|\log|V|)$ edges, and that
such a sparsifier can be found in $\tilde{O}(|E|)$ time. We finish by
discussing some applications of spectral sparsification, such as computing
electrical flows and finding minimum $s$-$t$ cuts.

\section{Introduction}

Sparse graphs have a roughly proportional number of edges and vertices, at
least up to a poly-logarathmic factor. In contrast, dense graphs (such as
the complete graph) permit a quadratic number of edges. Graph algorithms
usually run faster on sparse inputs, which conveniently also impose lower
space requirements. Moreover, edges in sparse graphs tend to have a larger
impact on the overall structure of the graph. Working with sparse graphs can
save time, save space, and provide insight into the nature of a graph: this
makes graph sparsification is interesting.

\section{Measuring Similarity}

We're trying to define `approximate'.

\subsection{Definitions}

\subsection{Cut similarity}

Def. cut-similar, discuss.

Describe Benczúr-Karger, state thm 1.

\subsection{Spectral similarity}

Def. spectrally-similar, discuss.

Describe Spielman-Teng, go up to thm 4.

\section{Sparsification with Sampling}

Cover the effective-resistances paper.

\section{Applications}

\section{Overview}

\section*{References}

\begin{enumerate}[1.]
    \item Achlioptas, D., Mcsherry, F. Fast computation of low-rank matrix
approximations. 2 (2007), 9.

    \item Batson, J.D., Spielman, D.A., Srivastava, N.  Twice-Ramanujan
sparsifiers. 6 (2012), 1704 –1721.

    \item Benczúr, A.A., Karger, D.R.  Approximating s-t minimum cuts in O(n 2
) time. In (1996), 47–55.

    \item Chandra, A.K., Raghavan, P., Ruzzo, W.L., Smolensky, R., Tiwari, P.
The electrical resistance of a graph captures its commute and cover times. In
(1989), ACM, New York, NY, USA, 574–586.

    \item Cheeger, J. A lower bound for smallest eigenvalue of Laplacian. In
(1970), Princeton University Press, 195–199.
\end{enumerate}

\end{document}
